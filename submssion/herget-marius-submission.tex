%%%%%%%%%%%%%%%%%%%%%%%%%%%%%%%%%%%%%%%%%
% Programming/Coding Assignment
% LaTeX Template
%
% This template has been downloaded from:
% http://www.latextemplates.com
%
% Original author:
% Ted Pavlic (http://www.tedpavlic.com)
%
% Note:
% The \lipsum[#] commands throughout this template generate dummy text
% to fill the template out. These commands should all be removed when
% writing assignment content.
%
% This template uses a Perl script as an example snippet of code, most other
% languages are also usable. Configure them in the "CODE INCLUSION
% CONFIGURATION" section.
%
%%%%%%%%%%%%%%%%%%%%%%%%%%%%%%%%%%%%%%%%%

%----------------------------------------------------------------------------------------
%	PACKAGES AND OTHER DOCUMENT CONFIGURATIONS
%----------------------------------------------------------------------------------------

\documentclass{article}

\usepackage{fancyhdr} % Required for custom headers
\usepackage{lastpage} % Required to determine the last page for the footer
\usepackage{extramarks} % Required for headers and footers
\usepackage[usenames,dvipsnames]{color} % Required for custom colors
\usepackage{graphicx} % Required to insert images
\usepackage{placeins}
\usepackage{listings} % Required for insertion of code
\usepackage{courier} % Required for the courier font
\usepackage{lipsum} % Used for inserting dummy 'Lorem ipsum' text into the template
\usepackage[utf8]{inputenc}
% \usepackage[shortlabels]{enumerate}


\usepackage{amsmath}
\usepackage{amssymb}

    \usepackage[T1]{fontenc}
\usepackage{enumitem}
\usepackage{tikz}
\usepackage{pgfplots}
\pgfplotsset{compat=1.13}

\pgfplotsset{
        discard if not/.style 2 args={
            x filter/.append code={
                \edef\tempa{\thisrow{#1}}
                \edef\tempb{#2}
                \ifx\tempa\tempb
                \else
                    \def\pgfmathresult{NaN}
                \fi
            },
        },
    }
% Margins
\topmargin=-0.45in
\evensidemargin=0in
\oddsidemargin=0in
\textwidth=6.5in
\textheight=9.0in
\headsep=0.25in

\linespread{1.1} % Line spacing

% Set up the header and footer
\pagestyle{fancy}
\lhead{\hmwkClass} % Top left header
\chead{} % Top center head
\rhead{\firstxmark} % Top right header
\lfoot{\hmwkAuthorName} % Bottom left footer
\cfoot{} % Bottom center footer
\rfoot{Page\ \thepage\ of\ \protect\pageref{LastPage}} % Bottom right footer
\renewcommand\headrulewidth{0.4pt} % Size of the header rule
\renewcommand\footrulewidth{0.4pt} % Size of the footer rule

\setlength\parindent{0pt} % Removes all indentation from paragraphs

%----------------------------------------------------------------------------------------
%	CODE INCLUSION CONFIGURATION
%----------------------------------------------------------------------------------------

\definecolor{MyDarkGreen}{rgb}{0.0,0.4,0.0} % This is the color used for comments
\lstloadlanguages{Perl} % Load Perl syntax for listings, for a list of other languages supported see: ftp://ftp.tex.ac.uk/tex-archive/macros/latex/contrib/listings/listings.pdf
\lstset{language=Perl, % Use Perl in this example
        frame=single, % Single frame around code
        basicstyle=\small\ttfamily, % Use small true type font
        keywordstyle=[1]\color{Blue}\bf, % Perl functions bold and blue
        keywordstyle=[2]\color{Purple}, % Perl function arguments purple
        keywordstyle=[3]\color{Blue}\underbar, % Custom functions underlined and blue
        identifierstyle=, % Nothing special about identifiers
        commentstyle=\usefont{T1}{pcr}{m}{sl}\color{MyDarkGreen}\small, % Comments small dark green courier font
        stringstyle=\color{Purple}, % Strings are purple
        showstringspaces=false, % Don't put marks in string spaces
        tabsize=5, % 5 spaces per tab
        %
        % Put standard Perl functions not included in the default language here
        morekeywords={rand},
        %
        % Put Perl function parameters here
        morekeywords=[2]{on, off, interp},
        %
        % Put user defined functions here
        morekeywords=[3]{test},
       	%
        morecomment=[l][\color{Blue}]{...}, % Line continuation (...) like blue comment
        numbers=left, % Line numbers on left
        firstnumber=1, % Line numbers start with line 1
        numberstyle=\tiny\color{Blue}, % Line numbers are blue and small
        stepnumber=5 % Line numbers go in steps of 5
}

\lstset{language=C++,
        basicstyle=\ttfamily,
        keywordstyle=\color{blue}\ttfamily,
        stringstyle=\color{red}\ttfamily,
        commentstyle=\color{green}\ttfamily,
        morecomment=[l][\color{magenta}]{\#}
        numbers=left, % Line numbers on left
        firstnumber=1, % Line numbers start with line 1
        numberstyle=\tiny\color{Blue}, % Line numbers are blue and small
        stepnumber=2 % Line numbers go in steps of 5
}

% Creates a new command to include a perl script, the first parameter is the filename of the script (without .pl), the second parameter is the caption
\newcommand{\perlscript}[2]{
\begin{itemize}
\item[]\lstinputlisting[caption=#2,label=#1]{#1.pl}
\end{itemize}
}

%----------------------------------------------------------------------------------------
%	DOCUMENT STRUCTURE COMMANDS
%	Skip this unless you know what you're doing
%----------------------------------------------------------------------------------------

% Header and footer for when a page split occurs within a problem environment
\newcommand{\enterProblemHeader}[1]{
\nobreak\extramarks{#1}{#1 continued on next page\ldots}\nobreak
\nobreak\extramarks{#1 (continued)}{#1 continued on next page\ldots}\nobreak
}

% Header and footer for when a page split occurs between problem environments
\newcommand{\exitProblemHeader}[1]{
\nobreak\extramarks{#1 (continued)}{#1 continued on next page\ldots}\nobreak
\nobreak\extramarks{#1}{}\nobreak
}

\setcounter{secnumdepth}{0} % Removes default section numbers
\newcounter{homeworkProblemCounter} % Creates a counter to keep track of the number of problems

\newcommand{\homeworkProblemName}{}
\newenvironment{homeworkProblem}[1][Task \arabic{homeworkProblemCounter}]{ % Makes a new environment called homeworkProblem which takes 1 argument (custom name) but the default is "Problem #"
\stepcounter{homeworkProblemCounter} % Increase counter for number of problems
\renewcommand{\homeworkProblemName}{#1} % Assign \homeworkProblemName the name of the problem
\section{\homeworkProblemName} % Make a section in the document with the custom problem count
\enterProblemHeader{\homeworkProblemName} % Header and footer within the environment
}{
\exitProblemHeader{\homeworkProblemName} % Header and footer after the environment
}


\usepackage{dashbox}
\newcommand{\problemAnswer}[1]{ % Defines the problem answer command with the content as the only argument
\\[0.5em]\noindent\framebox[\columnwidth][c]{\begin{minipage}{0.98\columnwidth}#1\end{minipage}}\\[0.5em] % Makes the box around the problem answer and puts the content inside
}

\newcommand{\note}[1]{ % Defines the problem answer command with the content as the only argument
\\[0.5em]\noindent\dashbox[\columnwidth][c]{\begin{minipage}{0.98\columnwidth}#1\end{minipage}}\\[0.5em] % Makes the box around the problem answer and puts the content inside
}


\newcommand{\homeworkSectionName}{}
\newenvironment{homeworkSection}[1]{ % New environment for sections within homework problems, takes 1 argument - the name of the section
\renewcommand{\homeworkSectionName}{#1} % Assign \homeworkSectionName to the name of the section from the environment argument
\subsection{\homeworkSectionName} % Make a subsection with the custom name of the subsection
\enterProblemHeader{\homeworkProblemName\ [\homeworkSectionName]} % Header and footer within the environment
}{
\enterProblemHeader{\homeworkProblemName} % Header and footer after the environment
}

%----------------------------------------------------------------------------------------
%	NAME AND CLASS SECTION
%----------------------------------------------------------------------------------------



%----------------------------------------------------------------------------------------
%	TITLE PAGE
%----------------------------------------------------------------------------------------

\title{
\vspace{2in}
\textmd{\textbf{\hmwkClass}}\\
\textit{\today}
\vspace{0.5in}
}

\author{\textbf{\hmwkAuthorName}}
\date{} % Insert date here if you want it to appear below your name

%----------------------------------------------------------------------------------------

\newcommand{\tabitem}{~~\llap{\textbullet}~~}
\usepackage{ltablex} % mix out of tabularx and longtable
\newcolumntype{L}[1]{>{\raggedright\let\newline\\\arraybackslash\hspace{0pt}}m{#1}}
\newcolumntype{C}[1]{>{\centering\let\newline\\\arraybackslash\hspace{0pt}}m{#1}}
\newcolumntype{R}[1]{>{\raggedleft\let\newline\\\arraybackslash\hspace{0pt}}m{#1}}


\setlist[enumerate]{noitemsep,nolistsep}
\setlist[description]{noitemsep,nolistsep}
\setlist[itemize]{noitemsep,nolistsep}

\lstset{ %
    basicstyle=\footnotesize,       % the size of the fonts that are used for the code
    numbers=left,           % where to put the line-numbers
    numberstyle=\footnotesize,  % the size of the fonts that are used for the line-numbers
    stepnumber=10,           % the step between two line-numbers. If it is 1 each line will be numbered
    numbersep=5pt,          % how far the line-numbers are from the code
    backgroundcolor=\color{white},  % choose the background color. You must add \usepackage{color}
    showspaces=false,           % show spaces adding particular underscores
    showstringspaces=false,     % underline spaces within strings
    showtabs=false,         % show tabs within strings adding particular underscores
    frame=single,           % adds a frame around the code
    tabsize=2,              % sets default tabsize to 2 spaces
    captionpos=b,           % sets the caption-position to bottom
    breaklines=true,            % sets automatic line breaking
    breakatwhitespace=false,        % sets if automatic breaks should only happen at whitespace
    float=H,
    escapeinside={\%*}{*)}      % if you want to add a comment within your code
}


\providecommand{\ra}{\ensuremath{\rightarrow}}

% url
\usepackage{hyperref}


\newcommand{\hmwkTitle}{} % Assignment title
\newcommand{\hmwkClass}{Submission for PhD position in computational social sciences at ETH Zürich} % Course/class
\newcommand{\hmwkAuthorName}{Marius Herget} % Your name
\usepackage{booktabs}
\usepackage[toc,page]{appendix}
\usepackage{rotating}
\usepackage{pdflscape}


\usepackage[defernumbers=true,style=numeric,backend=biber,]{biblatex}
\nocite{*}
\addbibresource{Literatur.bib}

\begin{document}
\maketitle\tableofcontents\newpage
%
% \problemAnswer{
% identifierstyle
%   % Marius Herget\\
%   %  \\
%   % Email: me@mariusherget.de
% }
%
\begin{homeworkProblem}
    My first experience with research was within my practical training for my B.Sc. employer at the IBM Almaden Research Center in 2016. I experienced first hand the enthusiasm of the scientist in asking and pursuing groundbreaking questions. After these three months, I decided to follow my inner craving to change the world. Especially the scientific approach with no prejudices, religion, or political agenda matches my inner conviction of just listening to facts. These experiences convinced me to achieve the highest university degree. Within my life so far, I always set myself the standard to obtain and deliver the best possible outcome. Consequently, I would like to continue my career at one of the best universities in Europe. \\

    My ambitions for my Ph.D. position is a mix of a great team and flexibility. Exploring new areas, receiving founded advice, and the willingness to surpass oneself should be the daily business. Nevertheless, the essential aspect of any future career step is the working environment. The personal and professional climate has to be friendly and giving me a strong impetus. Research wise I am very open-minded where the journey goes. I love the challenge of familiarizing myself with new facts and theories.
\end{homeworkProblem}

\begin{homeworkProblem}
  As already mentioned in my motivation letter, I think my technical skills are a perfect match for your team. The most exciting aspect is the practical implementation of various technologies (especially computational science) on social and political systems. These complex, traditional systems have a massive impact on society, and a critical analysis is crucial to regulate them.\\

  Nevertheless, your chair is accomplishing quite exciting research in more detail \textit{multi-layered networks} and \textit{socio-technical systems}. The aspect of the data-driven analysis is an innovative approach with great potential. In more detail, the paper \textit{"Quantifying and suppressing ranking bias in a large citation network"}~\cite{Vaccario2017Aug} has aroused my interests. The recognized biases within papers ranking and, therefore, their "false" importance is an enormous disruption of an old standing tradition of the scientific world.\\
\end{homeworkProblem}

\begin{homeworkProblem}
  Since the dataset is limited to only one specific research field, I would like to ask the following research question as a base for my data analysis:\\

\begin{center}\textbf{Exist unnoticed differences in wording within various research groups of a similar research area?}\\
\end{center}


  My analysis would take several steps:
  \begin{enumerate}
    \item Identify research groups based on the connection of \emph{author}, \emph{affiliations} with references of their publication to other papers and their authors (\emph{citations}).
    \item Identify significant words within the \emph{title}, \emph{abstract}, and \emph{full text}. This includes a data cleanup of common words like \textit{is}, \textit{are}, \textit{they}, \textit{and}, etc.
    \item Connect both results of steps $1.$ and $2.$ to recognize which research group applies which wording (= wording bubbles).
    \item Classify word bubbles and review their significance.
    \item Examine significant words by connecting synonyms within the dataset.
    Compare wording bubbles and identify via step $5.$ how various research groups differ in phrasing and content.
  \end{enumerate}


  The goal of this analysis is to recognize whether scientists miss relevant research publications in consequence of their \textbf{wording filter bubble}.\\

  This could be quite interesting for identifying other important papers while minimizing research expenses. On the one hand, in a practical application, a reduction of reading the current state of the art can be quite impelling. On the other hand, a broadening of the horizon can enhance a publication's quality.
\end{homeworkProblem}


\begin{homeworkProblem}
  This task is centered around the fact of finding the most important member of the House. Firstly it is crucial to define what importance means in this context:\\

  An important person is an individual who has the most power within the House. There are several aspects to consider:
  \begin{description}[labelindent=1cm]
    \item[Money] An influential person needs financial stability and has a high budget.
    \item[Reelections] Besides financial stability, a politician depends on its position in his party and its district's population (voters). Another aspect is that a member of the House gains more importance with his experience and the broader network.
    \item[Bills] Another aspect besides the politicians standing is the importance of his sponsored bills. In this assignment, this is shrunk down to the number of cosponsors and their party. It is more challenging to receive cosponsors from the opposite party.
  \end{description}
  This is only a small sample of a lot of different aspects to recognize the most important person.\\

  The following pseudo equations give a quick overview how the importance is calculated
  % \begin{equations}
    % \begin{equation}
      \begin{align}
      Imp_{Bill} & = (1.1 * \frac{CS_{other Party}}{CS_{all}} + 0.9 * \frac{CS_{own Party}}{CS_{All}}) * Normalized(CS_{Total})  \\
      Imp_{Representative} & = \frac{1}{2} * Normalized(\sum Money_{Individual}) + \frac{1}{2} * \frac{Periods~in~House_{last~4}}{4}\\
      Importance & = \frac{1}{2} * Imp_{Representative} + \frac{1}{2} * Normalized(\sum_{Sponsor = \newline Individual} Imp_{Bill})\\
    \end{align}
    % \end{equation}
  % \end{equations}

To achieve the correct analysis, the given data had to be supplemented by \cite{ProPublica:bills} and \cite{ProPublica:expenses}.\\

\textbf{Edward R. Royce} is according to my analysis with a significant lead the most import member:
\begin{description}[labelindent=1cm]
  \item[Party] Republican
  \item[Bills in 115] $33$
  \item[Available Money] $7340372.91\$$
  \item[In House of Representatives since] 1993
  \item[Received cosponsorships] $1685$
  \item[Given cosponsorships] $212$
  \item[Cosponsorships from Democrats] $822$
  \item[Cosponsorships from Republicans] $864$
  \item[Importance$_{Representative}$] $0.713342$
  \item[Importance$_{Bill}$] $4.292423$
  \item[Importance] $0.856671$
\end{description}
\end{homeworkProblem}

\newpage
\addcontentsline{toc}{section}{References}
\printbibliography

\newpage
\begin{landscape}
  \begin{appendices}
    \begin{longtable}{llrrrrrrrrrr}
\caption{Top ten most important members}\\
\toprule
                Name &       Party &  Bills &  Received &  Given &        Money &  Reelection &  $Imp_{Rep}$ &    $CS_{D}$ &   $CS_{R}$ &  $Imp_{Bill}$ &       $Imp$ \\
\midrule
\endhead
\midrule
\multicolumn{12}{r}{{Continued on next page}} \\
\midrule
\endfoot

\bottomrule
\endlastfoot
Royce, Edward R. &  Republican &     33 &      1685 &    212 &   7340372.91 &        1.00 &   0.713342 &   822 &   864 &    4.292423 &  0.856671 \\
Cicilline, David N. &    Democrat &     50 &      1689 &    674 &  10138790.22 &        0.75 &   0.670829 &  1641 &    47 &    3.862105 &  0.785289 \\
Maloney, Carolyn B. &    Democrat &     41 &      1329 &    435 &  11664144.51 &        1.00 &   0.840792 &  1195 &   136 &    3.092045 &  0.780571 \\
DeLauro, Rosa L. &    Democrat &     38 &      1338 &    409 &  10629422.22 &        1.00 &   0.810292 &  1307 &    31 &    3.061348 &  0.761745 \\
Engel, Eliot L. &    Democrat &     42 &      1289 &    500 &   8936514.00 &        1.00 &   0.760391 &   956 &   333 &    3.090439 &  0.740183 \\
Paulsen, Erik &  Republican &     39 &      1328 &    203 &   7100318.85 &        1.00 &   0.706266 &   563 &   766 &    3.305533 &  0.738176 \\
Lee, Barbara &    Democrat &     45 &      1217 &    751 &  10749907.02 &        1.00 &   0.813843 &  1178 &    39 &    2.765735 &  0.729086 \\
Velazquez, Nydia M. &    Democrat &     43 &      1129 &    584 &  10383839.10 &        1.00 &   0.803053 &   949 &   186 &    2.653293 &  0.710593 \\
Reichert, David G. &  Republican &     22 &      1161 &    145 &   7412045.22 &        1.00 &   0.715455 &   655 &   515 &    3.024453 &  0.710029 \\
Roe, David P. &  Republican &     30 &      1140 &    309 &   9508300.44 &        1.00 &   0.777245 &   268 &   879 &    2.752337 &  0.709227 \\
\end{longtable}

\textit{$\qquad Imp$: Importance, $CS$: Cosponsors, $Rep$: Representative, $D$: Democrat, $R$: Republican, Money in \$}

  \end{appendices}
\end{landscape}

\end{document}
